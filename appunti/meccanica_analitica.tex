% !TeX root = meccanica_analitica.tex
\documentclass{article}

\usepackage[T1]{fontenc}
\usepackage[utf8]{inputenc}
\usepackage[a4paper, total={15.3 cm, 21.3 cm}]{geometry}
\usepackage{amsmath}
\usepackage{amssymb}
\usepackage{amsthm}
\usepackage{gensymb}
\usepackage{booktabs}
\usepackage{hyperref}
\usepackage{caption}
\usepackage{float}
\usepackage{graphicx}
\usepackage{subfig}
\usepackage{titlesec}
\usepackage{titletoc}
\usepackage{physics}
\usepackage{siunitx}
\usepackage[dvipsnames]{xcolor}

\usepackage{longtable}
\usepackage{tabularx}
\usepackage{calc}
\usepackage{array}
\usepackage{subfiles}
\usepackage{etoolbox}
\usepackage{xparse}

\newtheorem{theorem}{Theorem}[section]
\newtheorem{corollary}{Corollary}[theorem]
\newtheorem{lemma}{Lemma}[theorem]
\newtheorem{definition}{Definition}[section]
\newtheorem{example}{Example}[section]
\newtheorem{proposition}{Proposition}[section]

\hypersetup{colorlinks=true,linkcolor=black}
\renewcommand\thesection{\arabic{section}}
\titlecontents{chapter}[1.05em]{\bigskip}
{\contentslabel[\MakeUppercase{\romannumeral\thecontentslabel}]{1em}\enspace\textsc}
{\hspace*{-1em}\textsc}
{\hfill\contentspage}
\titlecontents{section}[1.6em]{\smallskip}
{\thecontentslabel.\enspace}
{}
{\titlerule*[1pc]{.}\contentspage}
\setcounter{tocdepth}{2}


\begin{document}

    \pagenumbering{roman}
    \thispagestyle{empty}

    % First page with uni logo and title
    \begin{center}

        \includegraphics[width=1.\linewidth]{unimi_logo.jpg}
        \centering
        \vspace{3cm}

        \uppercase{\Large APPUNTI DEL CORSO MECCANICA ANALITICA \par}
        \vspace{3cm}

        \Large Lorenzo Liuzzo\par
        \vspace{1.5cm}

        \Large \today

    \end{center}

    \clearpage

    % Table of contents
    \tableofcontents

    \clearpage
    \pagenumbering{arabic}

    \section{Equazioni di Lagrange}

        \subsection{Problema ad un corpo}
        
            \begin{theorem}[della forza viva]
                \label{thm:Forza viva}

                Sia $T$ la forza viva (o energia cinetica) come $T = \frac{1}{2}m \dot{x} \cdot \dot{x}$.
                Allora lungo ogni soluzione $x = x(t)$ dell'equazione di Newton $m \ddot{x} = F(x)$, si ha 
                \[ \dot{T} = F(x) \cdot \dot{x} \] o equivalentemente \[ T(t_1) - T(t_0) = \int_{t_0}^{t_1} F(x) \cdot \dot{x} \, dt \]

                dove $F(x) \cdot \dot{x}$ è la potenza della forza e $\int_{t_0}^{t_1} F(x) \cdot \dot{x}$ è il lavoro svolto dalla forza. 

            \end{theorem}
            \begin{proof}

                Moltiplicando l'equazione di Newton per $\dot{x}$ e applicando la regola di Leibniz per la derivata di un prodotto, si ottiene
                \[ \dot{x} \cdot (m \ddot{x}) = \frac{d}{dt}(\frac{1}{2} m \dot{x} \cdot \dot{x}) = \dot{T} \]

            \end{proof} 

            Nel caso di un campo di forze posizionali $F = F(x)$ l'integrale a secondo membro dipende dal movimento $x(t)$ nell'intervallo $(t_0, t_1)$ 
            attraverso la corrispondente traiettoia $\gamma$. Si ha dunque un integrale curvilineo della forma differenziale: 
            \[ T(t_1) - T(t_0) = \int_{\gamma} F(x) \cdot dx \]  \\
            Nel caso in cui la forza sia conservativa, cioè in cui la forza ammetta potenziale, ossia una funzione scalare $V = V(x)$ tale che $F = - grad(V)$, 
            chiamiamo $V$ energia potenziale (o funzione delle forze). Questa condizione può essere riscritta dicendo che la forma differenziale del lavoro 
            è esatta, cioè è il differenziale di una funzione, in particolare: $F(x) \cdot dx = -dV(x)$. \\

            \begin{theorem}[dell'energia]
                \label{thm:Conservazione dell'energia}

                Per un punto soggetto ad un campo di forze posizionali conservativo $F = F(x)$, lungo ogni soluzione $x = x(t)$ dell'equazione di Newton $m \ddot{x} = F$, 
                si ha \[ E = T + V \quad \dot{E} = 0 \]
                
            \end{theorem}
            \begin{proof}

                \[ F \cdot \dot{x} = - \frac{\partial V}{\partial x} \cdot \frac{dx}{dt} = - \frac{dV}{dt} = - \dot{V} \quad \Longrightarrow \quad \dot{T} = -\dot{V} \]
                oppure, sfruttando la definizione integrale di lavoro (\ref{thm:Forza viva}) e il fatto che l'integrale curvilineo di una forma differenziale esatta
                lungo una curva orientata dipenda solo dai suoi estremi (cioè che $\oint F(x) \cdot dx = 0$), si ha
                \[ \int_{\gamma}F(x) \cdot dx = - \int_{\gamma} dV = V(A) - V(B) \]

            \end{proof}
            
            \begin{definition}[costante del moto]
                \label{def:costante del moto}

                Una variabile dinamica che assume un medesimo valore per ogni punto del moto corrispondente ad una soluzione dell'equazione di Newton è detta \boldmath{costante del moto}.
            
            \end{definition}
            Dunque, la funzione $E(x, \dot{x}) = T(\dot{x}) + V(x)$ è una costante del moto. Pertanto, fissati i dati iniziali $x_0$ e $\dot{x}_0$, e quindi anche $E_0$,
            il teorema di conservazione dell'energia totale (\ref{thm:Conservazione dell'energia}) va intesa nella forma \[ T - V = E_0 = E(x, \dot{x})\] \\
            Introducendo la quantitità di moto $p = m \dot{x}$, è possibile riscrivere dell'equazione di Newton come \[ \dot{p} = F \quad \Longleftrightarrow \quad p(t_1) - p(t_0) = \int_{t_0}^{t_1} F dt \] 
            dove l'integrale della forza nel tempo viene detto impulso della forza. \\

            \begin{theorem}[del momento angolare]
                \label{thm:momento angolare}

                Per un punto materiale soggetto ad una generica forza $F = F(x, \dot{x}, t)$, lungo ogni soluzione $x = x(t)$ dell'equazione di Newton $m \ddot{x} = F$, si ha
                \[ \dot{L} = M \] dove $L = x \times p$ è il momento angolare e $M = x \times F $ è il momento angolare della forza.
                
            \end{theorem}
            \begin{proof}
                
                Moltiplicando vettorialmente l'equazione di Newton per $x$ e applicando la regola di Leibniz per la derivata di un prodotto, si ottiene
                \[ F \times x = x \times m \ddot{x} = x \times \frac{d}{dt}(m\dot{x}) = \frac{d}{dt}(x \times p) - \frac{dx}{dt} \times p = \frac{d}{dt}(x \times p) \]
           
            \end{proof}

            \begin{corollary}(conservazione del momento angolare)
                \label{cor:conservazione del momento angolare}

                Per un punto materiale soggetto ad un campo di forze centrali $F = F(x)$, lungo ogni soluzione $x = x(t)$ dell'equazione di Newton $m \ddot{x} = F$, si ha
                \[ \dot{L} = 0 \] ossia che il momento angolare $L$ è una costante del moto.

            \end{corollary}
            \begin{proof}

                Si applica il teorema del momento angolare (\ref{thm:momento angolare}). Poichè $F$ è un campo di forze centrali, si ha che la forza è parallela al raggio vettore,
                perciò $ M =  x \times F = 0 \Longrightarrow \dot{L} = 0 $.

            \end{proof}

            \begin{corollary}(campi centrali e moti bidimensionali)
                \label{cor:campi centrale e moti bidimensionali}

                Per un punto materiale soggetto ad un campo di forze centrali $F = F(x)$, per ogni soluzione $x = x(t)$ dell'equazione di Newton $m \ddot{x} = F$, 
                la traiettoia $x = x(t)$ giace in un piano passante per il centro di forza e ortogonale al vettore $L$ momento angolare, determinato dalle 
                condizioni iniziali $x_0$ e $\dot{x}_0$ che definiscono $L_0 = x_0 \times m \dot{x_0}$.

            \end{corollary}
            \begin{proof}
                
                Per le proprietà del prodotto vettoriale, si ha che $L = x \times p$ è ortogonale a $x$. Ma per la conservazione del momento angolare (\ref{cor:conservazione del momento angolare}),
                si ha che $L$ è una costante del moto. Dunque, per ogni tempo $t$, il vettore $x(t)$, dovendo essere ortogonale ad un vettore costante, giace in un piano ortogonale a quel vettore.

            \end{proof}

            Prendendo l'equazione della quantità di moto $\dot{p} = F$ e proiettandola su un asse qualsiasi, ad esempio $x$, si ottiene $\dot{p}_x = F_x$ e 
            quindi se $F_x = 0 \Longrightarrow \dot{p}_x = 0$ e quindi $p_x = cost$. \\
            Ma per forze posizionali conservative, $F_x = 0$ corrisponde a dire che $V$ è invariante per traslazione lungo l'asse $x$, ossia 
            \[ \frac{\partial V}{\partial x} = 0 \quad \textrm{ovvero che} \quad V(x + h, ...) = V(x, ...) \quad \forall h \] \\
            \begin{proposition}
                \label{prop:simmetrie e conservazioni}

                Se l'energia potenziale $V$ è invariante per traslazioni lungo un asse, allora la componente della quantità di moto $p$ lungo quell'asse è una costante del moto.

            \end{proposition}



    \clearpage
    \section{Equazioni di Hamilton}

    \clearpage
    \section{Principi variazionali}



\end{document}